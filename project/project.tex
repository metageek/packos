\documentclass{article}
\title{Operating Systems II, 91.516-201, Semester Project}
\author{John Stracke}
\date{12 May 2006}

\begin{document}

\maketitle

\abstract{I developed a user-space simulation of a microkernel, called
PackOS, to explore the idea of using IPv6 for all IPC.  The system is
working, with sufficient application-level functionality to support a
simple HTTP server.}

\section{Introduction}

Much work has been done on minimal
microkernels  in the past 15 years,
starting with Jochen Liedtke's L3.\cite{Liedtke:1993}
Liedtke's key insight was that first-generation microkernels such as
Mach, derived from monolithic kernels, had too much inherited baggage
to be efficient.  By starting from scratch with L3, he was able to
build a system that made better use of the virtues of microkernels.
L3, though, turned out to be still too inefficient, so Liedtke started
over with L4, focusing on high-speed IPC.\cite{Liedtke:AchievedIPC}

PackOS builds on the lessons of L4; it is a minimal microkernel,
offering only a handful of services.  Its key difference from L4 is
that the kernel's IPC services are simpler.  L4's IPC takes the form
of synchronous RPC, with structured messages that can include both
memory objects and inline byte strings; and its
``Clans and Chiefs''\cite{Liedtke:ClansAndChiefs}
model allows for sophisticated routing and filtering of messages.
PackOS IPC, on the other hand, is asynchronous, unrouted, and
best-effort--a virtual network rather than an RPC system.  All packets
in this kernel-area network (KAN) are IPv6; this makes it possible to
connect to outside services directly, in contrast with RPC-based IPC
systems, which require a gateway to tunnel RPC requests over the
external network.

\section{Design}

The PackOS kernel offers three services:

\begin{enumerate}
\item{The KAN.}
\item{Interrupt dispatch.}
\item{Context switching.}
\end{enumerate}

These services are all designed to be as lightweight as possible.  In
particular, there is no such thing as an outstanding kernel operation,
which means that there is no need for kernel threads, which means
that threading will be much cheaper than in systems where every thread
requires the allocation of a kernel-space stack.

In addition, all operations should be achievable in \(O(1)\)
time.\footnote{At present, the kernel performs linear searches for
packet delivery, which means performance is actually \(O(N)\), where N
is the maximum number of processes in the system.  However, a minor
redesign, with hash tables, will enable \(O(1)\) performance.} With a
guarantee that all kernel calls complete in \(N\) \(\mu\)s, it may
become feasible to use a big kernel lock even on an SMP
implementation, vastly simplifying kernel implementation.

\subsection{The KAN.}

All microkernels need high-speed IPC.\cite{Liedtke:AchievedIPC}
PackOS's IPC mechanism is the KAN, the
kernel-area network.  The KAN is a virtual link layer, specialized for
carrying IPv6 packets.  The KAN's link-layer addresses are IPv6
addresses, and the packets it carries are IPv6 packets prefixed with a
link-layer header (just the link-layer source and destination
addresses).  Packet delivery is zero-copy; all packets are
page-aligned, and the pages are unmapped from the sender and mapped
into the recipient.\footnote{As will be obvious, this model will pose
complications when multicast is added.  It is unclear at this time
what the best solution is.  One way would be to map the packet into
the recipients read-only, or COW, and maintain a reference count;
another would be to send all multicast packets to a user-space server,
which would then retransmit them to subscribers.} This provides the
efficiency of L4 messages consisting of pages, without the complexity
of L4's structured messages.  Finally, the fact that the KAN is
asynchronous means that PackOS should not be vulnerable to the risks
identified by Jonathon Shapiro.\cite{Shapiro:2003}

User-space applications do not normally call the kernel's KAN
functions directly; instead, they use user-space libraries which
expose UDP, or TCP functionality.  These libraries are in turn built
on the IP library, which makes it possible to implement network
interface classes to abstract away the differences between link
layers.  The IP library implements IPv6's network layer, such as IP
headers and ICMP messages.

\subsection{Context switching}

The PackOS kernel knows nothing about processes; processes are managed
by the user-space scheduler.  Instead, the kernel exposes facilities
for creating contexts, manipulating their TLBs, and switching to them;
the scheduler can use these facilities to provide abstractions such as
processes and threads.

There is some natural overlap between packet delivery and context
switching.  When sending a packet, the usual behavior is to yield the
CPU to the receiving context.  The scheduler is the main exception to
this behavior; if it yielded the CPU every time it sent a packet, its
job of scheduling processes would be complicated.

\subsection{Interrupt dispatch}

In PackOS, all drivers are in
user space, as in MINIX v3\cite{Tanenbaum:2006} and L4.  Drivers
register with the kernel to handle interrupts; when interrupts occur,
the kernel delivers them as packets.  In the present implementation,
this system is used for clock ticks and for a network interface.

Drivers need to be separated
into top and bottom halves, as in most OSes.\footnote{This separation
is not present in the current simulator.}
The bottom half
receives an interrupt, manages the device, extracts whatever extra
information the device is trying to send, clears the interrupt, and
sends a packet to the top half.  When it sends that packet, the kernel
takes that to mean that the interrupt has been handled, and it's
safe to reenable interrupts.  The top half, on the other hand,
handles requests from other processes, manipulates the device, and
responds to packets from the bottom half.

One driver which may reasonably profit from being placed in kernel
space is the clock.  MINIX v3 places its clock driver in the kernel;
after seeing the overhead of implementing my clock in user space, I
can understand why.  I may have to follow MINIX's lead in this
respect.

MINIX also offers a solution to an x86-specific problem: how to
perform I/O instructions.  The x86, with its pedigree rooted
in Intel's early
microcontrollers, has an I/O bus separate from the memory bus.  
The
I/O bus is not subject to the MMU, and
the I/O instructions are not available in user mode.
On the
PC platform, there are various devices that can be accessed only via
the I/O bus.  These are mostly the legacy devices, such as PS/2 and
serial ports; modern PCI devices rarely require use of the I/O bus.
However, those legacy devices include some important functionality;
if PackOS is to support them, drivers will need
to be able to access the I/O bus.  MINIX's solution is to provide
system calls to perform I/O instructions, with appropriate
access control.  This is a workable solution; it is clearly less
efficient than invoking I/O instructions directly,
but it's more secure--and the legacy devices that require the I/O bus
are all relatively low-speed, so inefficiency is less of a problem
than it would be for, say, a Gigabit Ethernet controller.

There may be another solution to the I/O problem.  PCI permits devices
to request I/O ports; but PCI is not only for x86, and most non-Intel
CPUs do not have an I/O bus.  For a CPU such as the PowerPC to use
PCI, its Northbridge, or its PCI controller, must translate
memory accesses into the I/O accesses expected by the PCI devices.
Some x86 motherboards may have similar capabilities; if so, PackOS
could run more efficiently by requesting similar translation.

\section{Achievements}

Currently, PackOS has:

\begin{enumerate}
\item{Core KAN functionality.}
\item{An API for implementing IPv6 interfaces.}
\item{An IPv6 API, based on the interface system.}
\item{An IPv6 interface based on the KAN.}
\item{An IPv6 interface for exchanging packets with the outside world
  via shared memory.}
\item{The Linux side of the shm-based network interface.}
\item{A router library, to route packets between the KAN and the
  outside world.}
\item{ICMP, UDP, TCP.}
\item{A basic user-space scheduler.}
\item{Interrupt handling.}
\item{A clock interrupt service.}
\item{A basic file access API, with filesystem drivers.}
\item{A filesystem driver for a simple file sharing protocol,
  DumbFS.\footnote{DumbFS is an extremely crude protocol; it runs over
  UDP, with fixed retransmission intervals.  I didn't have time to
  implement NFS, so I had to invent something; but I wanted to give it
  a name that would make it cry out to be replaced.}}
\item{A Linux-based DumbFS server.}
\item{An API to enable schedulers to provide timer services.}
\item{An API for requesting timer services.}
\item{A basic HTTP/1.0 server.}
\end{enumerate}

\section{The core PackOS API}

These are the elementary functions of PackOS.  Some are kernel calls;
some are library functions manipulate data structures to pass to the
kernel calls.

One piece of background: most PackOS functions take a pointer to a
PackosError.  It's analogous to the POSIX errno, except that it's not
a global variable.  There's a library function PackosErrorToString(),
analogous to POSIX strerror().  As in POSIX, the general custom is
that functions that return pointers flag an error by returning NULL,
while functions that return integers flag an error by returning a
negative number.  When neither is applicable (for example, if the
function can legitimately return negative numbers), the function
guarantees to set the error to packosErrorNone on success.  (Other
functions make no guarantee of the error value on success.)

This document will make reference to objects and subclasses; this is
an approximation.
Although PackOS is implemented in C, much of the API is more or less
object-oriented, using opaque struct pointers.  Some parts, such as
filesystems and network interfaces, even have a form of subclassing,
in which the functionality is implemented separately by backends.

\subsection{Context manipulation}

A context is a suspended processor state: basically, the stack,
registers, and TLBs.  The kernel does not manage contexts; that's up
to the scheduler.  However, the kernel does need a list of all active
contexts, in order to be able to deliver packets to them.

Contexts belong to the scheduler's memory space; they are constructed
and manipulated in user space.  The only context operation for which
the kernel is involved is context switching.

The scheduler is a privileged process, started at initialization,
which is able to perform certain extra operations on contexts, such as
switching to a specified context and manipulating context TLBs.

\subsubsection{Types}

A PackosContext object represents a context.

\subsubsection{System call: PackosContextYield()}

Yields the CPU to the scheduler.  This is the only context-related
system call available to processes other than the scheduler.

\subsubsection{System call: PackosContextYieldTo()}

Yields the CPU to the specified context.  Available only to the
scheduler.

\subsubsection{Library function: PackosContextNew()}

Used by the scheduler to construct a new context.

\subsubsection{Library function: PackosContextBlocking()}

Used by the scheduler to check whether a context is blocking (waiting
for a packet on the KAN) before yielding it the CPU.

\subsubsection{Library function: PackosContextFinished()}

Used by the scheduler to check whether a context has exited.

\subsection{Interrupt handling}

A driver that needs to monitor a particular interrupt registers with
the kernel, specifying the interrupt number and providing a UDP port
number.  When the interrupt occurs, the kernel will send a UDP packet
to that port, containing the interrupt number.

At this time, the interrupt system is a bit sketchy, since the current
implementation is a simulation.  In particular, there are no
facilities for controlling which processes can register for which
interrupts.  Clearly, that will be needed eventually.

\subsubsection{Types}

An InterruptId is an integer which identifies an interrupt.
InterruptIds are hardware-specific.

\subsubsection{System call: PackosInterruptRegisterFor()}

Used by a driver's bottom half to register for an interrupt.

\subsubsection{System call: PackosInterruptUnregisterFor()}

Used by a driver's bottom half to unregister for an interrupt.

\subsection{Network functions}

\subsubsection{Types}

PackosAddress represents an IPv6 128-bit address.  PackosAddressMask
represents an address mask, a tuple of (address, length).

PackosPacket represents a packet; it is an IPv6 packet with a KAN
header consisting of source and destination KAN addresses (which are
not necessarily the same as the source and destination addresses in
the IPv6 header, if
the packet is addressed to or from the outside world).

\subsubsection{Library function: PackosMyAddress()}

Returns the current context's IPv6 address on the KAN.

\subsubsection{Library function: PackosSysAddressMask()}

Returns the address mask for the KAN.

\subsubsection{Library function: PackosSchedulerAddress()}

Returns the KAN address of the scheduler process.

\subsubsection{Library function: PackosPacketGetKernelAddr()}

Returns the KAN address of the kernel.  At this time, there is no
reason to send packets to the kernel.  However, the kernel sends
packets to interrupt handlers, so it's a good idea for an interrupt
handler to double-check the source address of any incoming interrupt
packet.

\subsubsection{Library function: PackosAddrFromString()}

Parses an IPv6 address string.

\subsubsection{Library function: PackosAddrToString()}

Converts a PackosAddress to a string.

\subsubsection{Library function: PackosAddrEq()}

Tests two PackosAddresses for equality.

\subsubsection{Library function: PackosAddrMatch()}

Tests whether a PackosAddress lies within a given PackosAddressMask.

\subsubsection{Library function: PackosAddrIsMcast()}

Tests whether a PackosAddress is a multicast address.

\subsubsection{System call: PackosPacketAlloc()}

Allocates a packet.  A system call is required because packets are
allocated as pages, so that they can be remapped from sender to
receiver, instead of being copied.  However, for a performant
implementation, it is likely that some sort of pooling will be
required, to reduce the frequency of system calls.  Such pooling
should, of course, be transparent to the application programmer.

\subsubsection{System call: PackosPacketFree()}

Frees a packet allocated with PackosPacketAlloc().

\subsubsection{System call: PackosPacketSend(()}

Sends a packet.  The packet must have been allocated with
PackosPacketAlloc(), so that it's a separately mapped page, which can
be transferred to the recipient's memory map.

The call includes a flag indicating whether to yield the CPU to the
recipient.  Applications almost never call PackosPacketSend()
themselves; instead, they call the IP interface library, which
abstracts away the differences between the KAN and other interfaces.
If they need to specify the yield flag, they can do so on a
per-interface basis.  The per-interface default is to yield to the
recipient; but the scheduler sets the flag not to yield, because that
would interfere with its schedule.

\subsubsection{System call: PackosPacketReceive()}

Returns a packet received on the KAN; if none is available, blocks
until one arrives (or until an error occurs).  The packet received is
guaranteed to have been allocated with PackosPacketAlloc().

The kernel guarantees that the KAN source address has not been
spoofed.

\subsubsection{System call: PackosPacketReceiveOrYieldTo()}

Like PackosPacketReceive(); but, if it blocks, it yields to a
specified context (instead of the scheduler).  For use only by the
scheduler.

\subsection{Paging functions}

These functions manage a context's TLB.  They're not complete; I
designed them before we covered TLBs in class, which means that
they're missing some important capabilities, such as controlling write
access.  In any event, they aren't fully implemented in the PackOS
simulation.

Note that all these functions are library functions; they manipulate
the user-space data structures from which the TLBs are updated during
a context switch.  However, they are available only to the scheduler,
since that's the only process which has the PackosContexts.

\subsubsection{Types}

A PackosPageEntry represents an entry in the TLB (a range of pages).

\subsubsection{Library function: PackosPageEntryNew()}

Allocates a new PackosPageEntry; specifies the location and size in
physical memory, and whether or not it's writable.

\subsubsection{Library function: PackosPageEntryDelete()}

Deletes a PackosPageEntry.

\subsubsection{Library function: PackosPageEntryInstall()}

Installs a PackosPageEntry into a context's TLB.

\subsubsection{Library function: PackosPageEntryUninstall()}

Removes a PackosPageEntry from a context's TLB.

\subsubsection{Library function: PackosPageEntryTransfer()}

Moves a PackosPageEntry from one context's TLB to another's.

\subsubsection{Library function: PackosPageEntryGetLength()}

Gets the length of an entry.

\subsubsection{Library function: PackosPageEntryGetAddr()}

Gets the virtual address to which an entry is mapped.

\subsubsection{Library function: PackosPageAlloc()}

Allocates a new PackosPageEntry for a page-aligned range of physical
memory of a given size.

\subsubsection{Library function: PackosEntryOfPage()}

Looks up the PackosPageEntry to which a given virtual address
corresponds.

\section{Higher-level libraries}

In addition to the core API, there are various libraries which layer
additional functionality on top of the kernel (most notable IPv6
options, ICMP, UDP, and TCP).

\subsection{The IPv6 interface library}

The KAN system calls are adequate for the KAN, but PackOS needs other
network interfaces.  For example, the current simulation has a network
interface based on shared memory and POSIX signals; a hardware-based
implementation would need to interface to, say, PCI Ethernet devices.
The IPv6 interface library provides a common abstraction for these
various link layers (in this view, the KAN is a link layer),
permitting us to build higher-level protocols on top of IP without
worrying about the link layer.

\subsubsection{Types}

An IpIface object represents a network interface; it can be used to send and
receive packets.  The core IpIface library handles packet queueing;
the driver modules interface to the underlying link layer.

\subsubsection{IpIfaceNativeNew()}

Constructs a native IpIface, one that interfaces to the KAN link
layer.  It is not possible to construct more than one per process; the
second attempt will report an error.

\subsubsection{IpIfaceNativeSetYieldToRecipient()}

Given a native IpIface, sets the flag
indicating whether or not to yield the CPU to the recipient when
sending a packet.

\subsubsection{IpIfaceNativeSetNextYieldTo()}

Scheduler only: set the PackosContext to which to yield the CPU the
next time we block waiting for a packet on a native IpIface.

\subsubsection{IpIfaceShmNew()}

In the simulation, allocates an IpIface for talking over a shared
memory segment.

\subsection{The IP library}

The IP library is responsible for sending and receiving packets over
network interfaces.  It implements core IPv6 functionality such as
headers and ICMP, and also includes functions for manipulating
packets.

\subsubsection{Types}

An IpHeader is a union which contains the type of, and a pointer to,
an IPv6 header.  The various supported types of headers are defined in
<ip.h>.

An IpHeaderIterator is an iterator over the headers in a packet.

\subsubsection{IpMcastJoin()}

Join a multicast group.  Unimplemented in the simulation.

\subsubsection{IpMcastLeave()}

Leave a multicast group.  Unimplemented in the simulation.

\subsubsection{IpSendOn()}

Send a packet on a given interface.

\subsubsection{IpSend()}

Send a packet on an unspecified interface; the library determines the
best one to use.

\subsubsection{IpReceiveOn()}

Receive a packet on a given interface.

\subsubsection{IpReceive()}

Receive a packet on any interface.

\subsubsection{IpPacketFromNetworkOrder()}

Convert a packet's IPv6 headers from network order to host
order.  Does not include TCP options.

\subsubsection{IpPacketToNetworkOrder()}

Convert a packet's IPv6 headers from host order to network
order.  Does not include TCP options.

\subsubsection{IpGetDefaultRoute()}

Get the address of the default gateway (and the IpIface on which to
send to reach that address).

\subsubsection{IpSetDefaultRoute()}

Set the address of the default gateway (and the IpIface on which to
send to reach that address).

\subsubsection{IpHeaderIteratorNew()}

Returns an IpHeaderIterator for iterating over the IPv6 headers of a
given packet.

\subsubsection{IpHeaderIteratorDelete()}

Deletes an IpHeaderIterator.

\subsubsection{IpHeaderIteratorNext()}

Takes an IpHeaderIterator; returns a pointer to the next IpHeader in
the packet, or NULL if none.

\subsubsection{IpHeaderIteratorHasNext()}

Takes an IpHeaderIterator; returns true if IpHeaderIteratorNext()
would return non-NULL.

\subsubsection{IpHeaderIteratorRewind()}

Resets an IpHeaderIterator back to the first header in the packet.

\subsubsection{IpHeaderSeek()}

Takes a packet; returns a pointer to the first IpHeader of the given
type in the packet, or NULL if there is none.

\subsubsection{IpHeaderAppend()}

Appends an IpHeader to a packet.

\subsubsection{IpPacketData()}

Returns a pointer to the payload of the packet (after all the
headers).

\subsection{UDP}

\subsubsection{Types}

A UdpSocket is an object for sending and receiving UDP packets.

\subsubsection{UdpSocketNew()}

Constructs a new UdpSocket.

\subsubsection{UdpSocketClose()}

Closes a UdpSocket.

\subsubsection{UdpSocketBind()}

Binds a UdpSocket to a given port.  Semantics are analogous to POSIX
bind().

\subsubsection{UdpPacketNew()}

Constructs a packet with fields filled out for sending from a given
UdpSocket.

\subsubsection{UdpSocketSend()}

Sends a UDP packet from a given UdpSocket.

\subsubsection{UdpSocketReceive()}

Returns a UDP packet received on a given UdpSocket.

\subsubsection{UdpSocketReceivePending()}

Checks to see whether or not there's a packet in the receive queue of
a UdpSocket.

\subsection{TCP}

\subsubsection{Types}

A TcpSocket is an object for sending and receiving TCP packets.

\subsubsection{TcpSocketNew()}

Constructs a new TcpSocket.

\subsubsection{TcpSocketClose()}

Closes a TcpSocket.

\subsubsection{TcpSocketBind()}

Binds a TcpSocket to a given port.  Semantics are analogous to POSIX
bind().

\subsubsection{TcpSocketConnect()}

Connects a TcpSocket to a given remote address and port.  Blocks until
the connection succeeds (or fails).

\subsubsection{TcpSocketListen()}

Configures a TcpSocket to listen for connections.  At this time, there
is no analogue to the {\tt backlog} argument to POSIX's listen(2); the
backlog is fixed at a size of 1.

\subsubsection{TcpSocketAccept()}

Given a listening TcpSocket, blocks until there is an incoming
connection and returns a new TcpSocket for it.

\subsubsection{TcpSocketAcceptPending()}

Checks to see whether there's a pending connection; that is, whether
TcpSocketAccept() would return non-NULL without blocking.

\subsubsection{TcpSocketSend()}

Sends data on a connected TcpSocket.

\subsubsection{TcpSocketReceive()}

Receives data on a connected TcpSocket.  Blocking.

\subsubsection{TcpSocketReceivePending()}

Checks to see whether there is pending data to receive on a connected
TcpSocket.

\subsubsection{TcpPoll()}

Gives the TCP stack a chance to check for packets.  Shouldn't be
necessary, but it is, for now.

\subsection{File I/O}

An API, very crude for now, for reading and writing files.  There is
a filesystem abstraction layer; filesystem implementations do the work
of file I/O, and construct File objects which represent open files.
At present, it is necessary to open each filesystem explicitly; an
obvious improvement would be to create an overall filesystem with
mounting operations.

At present, there is only one filesystem, implementing a crude file
sharing protocol called DumbFS.

\subsubsection{Types}

A FileSystem object represents a filesystem; a File object represents
an open file.  Basically, a FileSystem defines or more more subclasses
of File, and acts as a File factory.

\subsubsection{FileSystemDumbFSNew()}

Constructs a FileSystem that speaks the DumbFS protocol.

\subsubsection{FileSystemClose()}

Closes down a FileSystem.

\subsubsection{FileOpen()}

Given a FileSystem, a path, and read/write flags, opens a file.

\subsubsection{FileClose()}

Closes a file.

\subsubsection{FileRead()}

Reads from a file; semantics similar to POSIX read(), except all
operations are blocking.

\subsubsection{FileWrite()}

Writes to a file; semantics similar to POSIX write(), except all
operations are blocking.

\subsection{Timer API}

Used to register for timer events, sent by the scheduler (since the
scheduler receives clock ticks from the kernel).  Timer notifications
are sent over UDP.  Used by TCP and DumbFS to manage their
retransmissions.

There is one problem with this system: since TCP relies on timers, the
timer control protocol can't use TCP.  More generally, it can't
involve any sort of timeout-based retransmissions (bootstrap problem).
I'm sure there's some sort of solution for this; perhaps the scheduler
can just assume that every process wants a timer for controlling the
timer control protocol.

\subsubsection{Types}

A Timer object represents a timer in use.

\subsubsection{TimerNew()}

Takes a UdpSocket (to which notifications will be sent), the numbers
of seconds and microseconds between ticks, an arbitrary ID number to
include in the notification packets, and whether or not to repeat the
ticks (if not, only one will be sent).  Returns a new Timer.

\subsubsection{TimerClose()}

Shuts down a given Timer; asks the server to stop sending
notifications.

\section{Limitations}

Not all of PackOS's capabilities are full-featured.  The following
limits remain:

\begin{enumerate}
\item{The IPv6 stack does not support fragments,
routing headers, or multicast.}
\item{The ICMP library
does not support all ICMP features, and is not fully connected to the
TCP library.}
\item{The paging system is essentially unimplemented.  The simulator
  has the beginning of an implementation based on mmap(); but it was
  put off in order to complete the TCP/IP stack on time.}
\item{There is no support for threads (not
much point, in the absence of memory protection).}
\item{The basic scheduler doesn't have a concept of process priority.
  As a result, the idle process gets a timeslice even when other
  processes need the CPU.}
\item{The file API
supports only open, read, write, and close; there's nothing like seek
(although the DumbFS protocol could support it), delete, rename, or
stat.}
\item{The DumbFS protocol is, er, named honestly.
(It was written before the TCP implementation,
when I wasn't sure I would be able to get TCP done in time.)  I
should implement NFS instead.}
\end{enumerate}

More seriously, the TCP implementation appears to have some bugs which
mean that the HTTP server cannot serve up any file that won't fit into
a single TCP segment (along with its HTTP headers).  When a client
requests a file that's too large, the client gets no data (although
tcpdump shows that the data is sent), and the HTTP server enters some
sort of invalid state, where it cannot answer any further requests.
For this reason, the sample pages included are quite small.

Finally, there are pervasive performance problems; all lookups are
linear searches.  Due to this overhead (and due to the fact that it
was developed largely on relatively slow machines), the system clock
runs at a resolution of 1/10 of a second.

\section{Future plans}

I intend to develop PackOS further, and submit it as a master's
thesis.  My future development plans include:

\begin{enumerate}
\item{Cleanup work:
\begin{enumerate}
\item{Fix the bug(s) in TCP.  This one is most critical;
  until I can get large chunks of data in and out reliably, it will be
  difficult to go further.}
\item{Implement missing features from IPv6 and ICMP.}
\item{Implement TCP Slow Start and the LFN extensions.}
\item{Implement TCP Path MTU Discovery.}
\item{Provide a single unified ``send a packet and wait to receive a
  packet'' operation, like L4,
  to cut down on context switching
  overhead.}
\item{Tackle the performance problems.  One significant step would be
  to remove some of the diagnostic messages I now longer need.  More
  seriously, I intend to modify the data structures and algorithms to
  ensure that all kernel operations can complete in \(O(1)\) time.}
\item{Expand the filesystem API to include richer functionality.}
\item{Implement a configuration protocol.}
\item{Implement a routing protocol.  At present, the IP layer
  has a concept of a default route, but that's it.  Also, the KAN
  address of the router is made available to other processes via a
  global variable, which is a hack that won't work in the presence of
  protected memory.}
\item{Replace DumbFS with something sensible.}
\end{enumerate}
}

\item{New development:
\begin{enumerate}
\item{Provide smoother APIs for the UDP and TCP libraries.  Something
  akin to select() would help a lot.}
\item{Port it to real hardware.  x86 is the obvious initial candidate,
  although I might try PowerPC and/or ARM as well.}
\item{Implement memory protection.}
\item{Implement paging.}
\item{Implement threading.}
\item{Provide a POSIX layer.}
\end{enumerate}
}
\end{enumerate}

\section{Lessons Learned}

The primary lesson learned here is that the basic idea of PackOS is
feasible; it is possible to build an IPv6-based microkernel which is
capable of supporting at least one real application.

One point in which PackOS's current design is insufficiently general
is that contexts and KAN addresses are coupled.  For greater
flexibility, it may be best to decouple them, so that multiple
contexts can share an address (for threading), or one context can have
multiple addresses (for Mobile IP forwarding).

Further, the kernel is quite small, only about 2650 lines, which
compiles to about 26K (on PowerPC).  Of course, the final kernel may
be somewhat larger, but I would be surprised to see it increase by as
much as a factor of two, since the current kernel supports all of the
planned functionality except TLB management and x86 I/O instructions.
Combined with the kernel's low memory utilization (due largely to the
absence of kernel threads), this means that a very small amount of
memory will have to be reserved for the kernel, perhaps as little as
64K.  (On x86, at least, there will have to be stacks for use in
exceptions, which will add to the total.) This will mean more memory
available to applications; and, by using fewer TLBs, may reduce the
cost of context switches.

\begin{thebibliography}{10}
\bibitem{Liedtke:1993}
Jochen Liedtke,
\newblock{Improving IPC by Kernel Design},
\newblock{appeared at 14th SOSP, 1993}

\bibitem{Liedtke:ClansAndChiefs}
Jochen Liedtke,
\newblock{Clans \& Chiefs},
\newblock{Architektur von Rechensystemen, 1991}

\bibitem{Liedtke:AchievedIPC}
Jochen Liedtke, Kevin Elphinstone, Sebastian Sch\"onberg, Hermann
H\"artig, Gernot Heiser, Nayeem Islam, Trent Jaeger,
\newblock{Achieved IPC Performance (Still the Foundation for
  Extensibility)},
\newblock{HotOS IV, 1997}

\bibitem{Shapiro:2003}
Jonathan S. Shapiro,
\newblock{Vulnerabilities in Synchronous IPC Designs},
\newblock{IEEE Symposium on Security and Privacy, 2003}

\bibitem{Tanenbaum:2006}
Andrew Tanenbaum,
\newblock{Can We Make Operating Systems Reliable and Secure?},
\newblock{IEEE Computer, May 2006, pp. 44-51}

\end{thebibliography}

\end{document}
